\section{Contexte}
\label{context}

\subsection{Sous-contexte}

\textbf{Bold} or \textit{Italic} or \textsc{small capitals}

This is a reference\footnote{Here is in passing how to make a footnote} to the \iref{context} section thanks to \codeinline{label} and \codeinline{iref} commande

\hfill

Use \codeinline{enumerate} in order to create a numbered list

\begin{enumerate}
    \item A
    \item B
\end{enumerate}

We can also use points lists

\begin{itemize}
    \item A
    \item B
\end{itemize}

A small link to my \href{https://github.com/av1m}{Github}

I can quote Dennis Ritchie\cite{c-programming}, or Olivier Cailloux\cite{cailloux}

Also, we can create figures or paintings

\begin{figure}[H]
    \centering
	\includegraphics[scale=1,width=0.5\textwidth]{images/dauphine.png}
	\caption{Logo from Paris Dauphine PSL University}
	\footnotesize A little note
	\label{dauphine-logo}
\end{figure}


\begin{table}[h]
    \centering
    \begin{tabular}{|l|c|r|}
      \hline
      \textbf{Left} & \textbf{Right} \\
      \hline
      Line 1 left & line 1 right \\
      Line 2 left & line 2 right \\
      \hline
    \end{tabular}
    \caption{A simple example of a table}
    \label{tab:duree}
    \footnotesize Another little note
\end{table}


